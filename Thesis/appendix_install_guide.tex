\section{Inštalačná príručka} \label{appendix_install_guide}

Pre spustenie verzie robota so serverom je potrebné mať nainštalované nasledovné programy. Opísaný postup fungoval na Windows 7 SP1 32 bit\footnote{postup inštalácie pre ostatné operačné systémy \url{http://simspark.sourceforge.net/wiki/index.php/Main_Page}}.

\subsection{Inštalácia servera}
\begin{enumerate}
	\item Inštalácia Microsoft Visual C++ 2008 Redistributable Package 
	\\ \url{http://www.microsoft.com/en-us/download/details.aspx?id=29}
	\item Inštalácia SimSpark
	\\ testované na verzii 0.2.4
	\\ \url{http://sourceforge.net/projects/simspark/files/simspark/}
	\item Inštalácia RcssServera
	\\ testované na verzii 0.6.7 
	\\ \url{http://sourceforge.net/projects/simspark/files/rcssserver3d/}
	\item Inštalácia Ruby
	\\ testované na verzii 1.9.3-p551
	\\ \url{https://www.ruby-lang.org/en/downloads/}
	\item Niekedy je potrebné reštartovať systém.
	\item Nastavenie premenných prostredia
	\begin{itemize}
		\item vytvoriť alebo pridať do premennej prostredia \texttt{PATH} cestu k inštalácii Ruby (napr. \texttt{C:\textbackslash Program Files\textbackslash ruby\textbackslash bin})
		\item vytvoriť premennú prostredia \texttt{SET SPARK\_DIR} a priradiť cestu k inštalácii (napr. \texttt{C:\textbackslash Program Files\textbackslash simspark})
		\item vytvoriť premennú prostredia \texttt{SET RCSSSERVER3D\_DIR} a priradiť cestu k inštalácii (napr. \texttt{C:\textbackslash Program Files\textbackslash rcssserver3d 0.6.7})
	\end{itemize}
\end{enumerate}

\subsection{Inštalácia hráča}
\begin{enumerate}
	\item Mať nainštalované prostredie Java, minimálne verziu 1.6
	\\ \url{http://www.oracle.com/technetwork/java/javase/downloads/index.html}
	\item Mať nainštalované prostredie Ruby.
	\item Importovať zdrojové súbory hráča
	\\ minimálne projekt \texttt{Jim} a \texttt{RobocupLibrary}
	\\ zdrojové súbory sú dostupné na elektronickom médiu (viď príloha \ref{appendix_medium}) alebo importovať z GitHub\footnote{\url{https://github.com/PaulNoth/robocup-fiit}}
	\\ najlepšie je využiť vývojové prostredie Eclipse\footnote{\url{http://www.eclipse.org/downloads/}} (testované aj na verzii Luna 4.4)
\end{enumerate}

\subsection{Spustenie hráča}
\begin{enumerate}
	\item Spustenie servera
	\item Spustenie monitora\footnote{alternatíva je použitie RoboViz \url{https://sites.google.com/site/umroboviz/}}
	\item Spustenie hráča
	\\ na ihrisku by sa mal objaviť Nao robot pred výkopom v strede ihriska
\end{enumerate}
Aktuálne zdrojové súbory majú nastavený plán pre vykonanie dynamického pohybu a stačí zapnúť simulačné prostredie.