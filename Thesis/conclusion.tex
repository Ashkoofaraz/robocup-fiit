\section{Zhodnotenie a ďalšia práca}

Na základe analyzovaných riešení sme vytvorili prototyp, ktorý dokáže využiť inverznú kinematiku a vypočítať veľkosti uhlov v končatinách, aby sa dostal na želanú pozíciu. Implementáciu inverznej kinematiky sme sa snažili urobiť čo najviac nezávislú od existujúceho kódu. Vznikol nový Java balíček.

Urobili sme tak základ pre vytvorenie nového spôsobu vytvárania pohybov. To umožní vytvoriť nové pohyby, príp. nahradiť niektoré staticky definované pohyby v XML súboroch novými dynamicky definovanými.

Pohyby robota fungujú. Dokážeme vypočítať natočenie uhlov, aby sa koncový efektor dostal na želanú pozíciu. Ako bolo spomenuté, pohyby sú nestabilné. V ďalšej práci je potrebné doplniť vytvorenie pohybu takým spôsobom, aby robot nepadal. Pokúsime sa využiť existujúce riešenia opísané v analýze.

Ďalšou možnostou je nájsť spôsob, ako zafixovať hodnoty niektorých uhlov tak, aby sa efektor dostal na želanú pozíciu. Momentálne sú výpočty závislé na predošlom výpočte a neexistuje vyjadrenie rovnice hodnoty uhla opačným spôsobom.

Začali sme pracovať s implementáciou od Pavla Meštaníka (kapitola \ref{sec_mestanik}), ktorá ešte využívala Ruby skriptovanie. V súčasnej implementácii fakultného robota, ktorú vyvíjajú na tímových projektoch\footnote{Tím 8 Infinity - v akademickom roku 2014/2015 \url{http://labss2.fiit.stuba.sk/TeamProject/2014/team08is-si/} }, boli odstránené časti kódu v Ruby a plne nahradené Javou. Ďalším cieľom je pripojiť sa k hlavnej vetvy vývoja na fakulte k repozitáru tímového projektu a nefragmentovať funkcionalitu. Predpokladáme, že integrácia kódu z tejto diplomovej práce a kódu tímového projektu prebehne s menšími komplikáciami, pretože vytvorený prototyp má málo závislosti na iné časti kódu.

%Cieľom tejto práce bolo oboznámiť sa s problematikou simulovanej robotickej 3D ligy na Fakulte informatiky a informačných technológií. Zameriavame sa v nej na opis aktuálneho stavu riešenia fakultného hráča JIM. Jeho kvality stále nedosahujú úroveň najlepších svetových tímov. 
%V súčasnosti sa pre definovanie pohybov robota používajú XML súbory so staticky definovanými hodnotami natočenia kĺbov. Ich nevýhodou je, že pohyby sú vždy rovnaké a každý pohyb si vyžaduje presné začiatočné podmienky pred svojim vykonaním. 
%Jedným z možných rozšírení fakultného hráča, na ktoré sme sa zamerali, je vytvorenie dynamického a parametrizovateľného kopu. Súčasné kopy sú rovnaké a navyše kop pri inom nastavení tela hráča, ako je pozícia lopty oproti želanému smeru kopu, si vyžaduje aj prípravnú fázu, počas ktorej sa musí robot postaviť na správne miesto. Zároveň prípravná fáza zaberá dôležitý čas, počas ktorého môže súper sa dostať do vhodnejšej pozície k lopte, prípadne zablokovať strelu. 
%V práci opisujeme riešenie dynamického kopu svetovými tímami. A na záver pridávame opis jedného z algoritmov jedného z tímov, ktorý by mohol byť využiteľný pri vytváraní takého kopu pre fakultného hráča. Opis je doplnený o súčasné poznatky a časti implementácie zo starších prác vykonaných na hráčovi JIM, o ktorých si myslíme, že by mohli poslúžiť ako podklad pre riešenie dynamického kopu.