\section{Zhodnotenie a ďalšia práca} \label{sec_conclusion}

Táto práca sa zameriava na simulovanú robotickú 3D ligu autonómnych robotov. Vychádza z aktuálneho stavu implementácie hráča na Fakulte informatiky a informačných technológií. Na základe aktuálneho stavu hráča sme sa rozhodli rozšíriť schopnosti kopania.

Aktuálne sa dokáže hráč kopať na základe staticky definovaných pohybov v \texttt{XML} súboroch. Pohyby sú rovnaké, a súčasne pri inom ako želanom postavení hráča dochádza k nevyhovujúcemu nastaveniu pozície hráča. Spôsobené zdržanie môže byť využité súperom na odobratie držanej lopty.

Jedným z rozšírení kopacích schopností hráča je vytvorenie, resp. rozšírenie existujúceho staticky definovaného kopu o variabilné časti. Našli sme závislosť medzi natočením bedrového kĺbu, ktorý pohybuje dolnou kočatinou v smere osi $y$ ihriska a uhlom medzi priamym rovným kopom a konečnej pozície lopty. Na základe závislosti sme vytvorili taký kop do strany, ktorého na vstupe je želaný uhol konečnej pozície lopty a hráč súčasne nataví bedrový kĺb tak, že konečná pozícia smeruje do strany.

Ďalším novým kopom je kop na základe existujúceho pohybu. Zmenou je kopnutie do strany na základe postavenia hráča oproti lopte. Pri posune hráča sme objavili závislosť medzi kĺbami, ktoré sa podieľajú na posune. Natočenie kĺbu je možné vypočítať na základe vzdialenosti posunu.

Posledné dva kopy vznikli spojením predchádzajúcich dvoch pohybov. Hráč sa zároveň posunie do strany o želanú vzdialenosť s zároveň sa zafixuje bedrový kĺb. Týmito kopmi sme dosiahli najväčšie rozpätie uhla kopu do strany. Uhly sa nachádzajú približne v intervale $<-45;20>$ stupňov z pohľadu hráča.

Fakultnému hráčovi sme vytvorili nový modul doprednej a inverznej kinematiky, ktorý mu umožňuje vypočítať natočenie kĺbov na určenú pozíciu v priestore. V práci uvádzame, že pri testovaní modulu sa vyskytli problémy, že výpočty niektorých natočení kĺbov nie sú presné. Preto sme sa vytváraniu pohybov pomocou kinematiky nevenovali.

Možným rozšírením práce je nájdenie ďalších závislostí medzi kĺbami končatín vo vybranom pohybed. Ďalšou možnosťou je skúmanie vplyvu kĺbu \texttt{LLE3}, resp. \texttt{RLE3} na silu našich novovytvorených pohybov. Veľkou výzvou je pokus o objavenie chyby v použitých rovniciach doprednej kinematiky a následné zakomponovanie do fakultného hráča a vytvorenie úplne dynamického pohybu.