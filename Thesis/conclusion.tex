\section{Zhodnotenie a ďalšia práca} \label{sec_conclusion}

Na základe analyzovaných riešení sme vytvorili prototyp pre fakultného hráča, ktorý dokáže využiť inverznú kinematiku a vypočítať veľkosti uhlov v končatinách, aby sa efektor dostal na želanú pozíciu. Implementáciu inverznej kinematiky sme sa snažili urobiť čo najviac nezávislú od existujúceho kódu. Vznikol nový Java balíček.

Urobili sme tak základ pre vytvorenie nového spôsobu vytvárania pohybov. To umožní vytvoriť nové pohyby, príp. nahradiť niektoré staticky definované pohyby v XML súboroch novými dynamicky definovanými.

Pohyby robota fungujú. Dokážeme vypočítať natočenie uhlov, aby sa koncový efektor dostal na želanú pozíciu. Pohyby pre dolné končatiny sú momentálne nestabilné. V ďalšej práci je potrebné doplniť vytvorenie pohybu takým spôsobom, aby robot nepadal. Pokúsime sa využiť existujúce riešenia opísané v analýze.

Ďalšou možnosťou je nájsť spôsob, ako zafixovať hodnoty niektorých uhlov tak, aby sa efektor dostal na želanú pozíciu. Momentálne sú výpočty závislé na predošlom výpočte a neexistuje vyjadrenie rovnice hodnoty uhla opačným spôsobom.

Začali sme pracovať s implementáciou od Pavla Meštaníka (kapitola \ref{sec_mestanik}), ktorá ešte využívala Ruby skriptovanie. V súčasnej implementácii fakultného robota, ktorú vyvíjajú na tímových projektoch\footnote{Tím 8 Infinity - v akademickom roku 2014/2015 \url{http://labss2.fiit.stuba.sk/TeamProject/2014/team08is-si/} }, boli odstránené časti kódu v Ruby a plne nahradené Javou. Ďalším cieľom je pripojiť sa k hlavnej vetve vývoja na fakulte k repozitáru tímového projektu a nefragmentovať funkcionalitu. Predpokladáme, že integrácia kódu z tejto diplomovej práce a kódu tímového projektu prebehne bez väčších komplikácií, pretože vytvorený prototyp má málo závislosti na iné časti kódu.