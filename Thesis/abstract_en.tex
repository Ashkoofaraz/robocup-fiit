\begin{titlepage}
\large \textbf{Annotation} \\ \\
\normalsize
Slovak University of Technology Bratislava \\
FACULTY OF INFORMATICS AND INFORMATION TECHNOLOGIES \\
Degree Course: SOFTWARE ENGINEERING  \\
 \\
Author: Bc. Pavol Pidanič \\
Diploma Project: High skills of a simulated robotic soccer player \\ 
Supervisor: Ing. Ivan Kapustík  \\
2014, May \\
\\
The main topic of this thesis is world initiative RoboCup. RoboCup's goal is to support a development of a robotics and artificial intelligence. Simulated 3D robotic soccer league is part of RoboCup initiative and it is in the aim of researchers at Faculty of informatics and information technologies. In the beginning we start with the description of the faculty robot, named JIM and the actual achieved progress. The aim of this thesis is a problem of dynamic robot kicking. There are descriptions of the best robotic soccer teams and teams intended to dynamic and parametric kicking. We described an algorithm of one of the teams and knowledge related to dynamic kicking from faculty robot player.

%Práca je venovaná svetovej iniciatíve RoboCup, ktorej cieľom je podpora rozvoja robotiky a umelej inteligencie. Simulovaná robotická 3D liga je časť tejto problematiky, ktorej sa venujú aj výskumníci na Fakulte a informatiky a informačných technológií. Na začiatku opisujeme fakultného hráča s menom JIM a stav riešenia za posledné roky diplomovými prácami a na predmete Tímový projekt. Práca sa zameriava na problém riešenia dynamického kopu. Obsahuje opis riešení najlepších svetových tímov a tímov, ktoré sa venujú hlavne dynamickému a parametrizovateľnému kopu. V popise návrhu uvádzame algoritmus implementácie takého kopu jedného zo svetových tímov a poznatky získané z aktuálneho stavu fakultného hráča súvisiace s návrhom dynamického kopu.

\end{titlepage}