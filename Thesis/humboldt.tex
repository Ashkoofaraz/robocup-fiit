\subsubsection{Nao Team Humboldt} \label{humboldt}
Tím z berlínskej Humboldtovej univerzity sa venuje aj prispôsobovaniu pohybov na základe podmienok počas hry. V práci \cite{humboldt} sa zamerali na vytvorenie dynamicky prispôsobovanému kopu.

Ich kop pozostáva z 3 aspektov - dosiahnuteľný priestor, naplánovanie pohybu a stabilizovanie. Pohyb je modelovaný v karteziánskom priestore a natočenie kĺbov počítajú na základe inverznej kinematiky. Aby predišli riešeniu zložitého optimalizačného problému, rozhodli sa, že každá nová pozícia kĺbov sa vypočíta v samostatnom výpočtovom cykle.

Plánovanie pohybu kopu si rozdelili na 4 fázy - príprava na kop, odtiahnutie nohy, vykonanie kopu a prinoženie. Počas prípravnej fázy robot nastaví telo tak, aby stál na jednej nohe a nadvihol druhú. Pri odťahovaní nohy si určili funkciu, pri ktorej sa berie do úvahy bod, ktorý je potrebné dosiahnuť, aby nastal zásah do lopty a vektor, ktorý zodpovedá želanému smeru kopu. Výpočet tohto bodu zjednodušili na dvojrozmerný prípad. Súradnicu $z$, ktorá predstavuje výšku chodidla, určili ako konštantu, ktorou je priemer veľkosti lopty. Ak majú určený najlepší bod odtiahnutia, následne sa vykoná pohyb. Dráhu pohybu opisujú veľmi jednoduchým spôsobom. Trajektória predstavuje najkratšiu cestu v priestore medzi bodom odtiahnutia a loptou. Nejde však o najrýchlejší pohyb. Ale rozhodli sa tak preto, lebo pri najrýchlejšom pohybe môže nastať kontakt so zemou. Po odkopnutí lopty sa robot vráti do pôvodnej pozície pred kopom - štvrtá fáza. Ak by sa podmienky počas vykonávania zmenili natoľko, že nie je možné dokončiť kop, robot preruší plánovanie a pokračuje fázou prinoženia do začiatočnej pozície.

Dosiahnuteľný priestor zodpovedá priestoru, ktorý je možné dosiahnuť kopajúcou nohou, zatiaľ čo robot stojí stabilne na druhej nohe. Problém si zjednodušili tak, že neuvažovali natáčanie kĺbov na hraniciach priestoru. Takto im vznikla len trojrozmerná mriežka bez oblúkov. Tento priestor si vypočítali experimentálne na fyzickom robotovi Nao.

Stabilitu vyrovnávajú počas celého pohybu. Pri naplánovaní kopu sú dané pozície oboh nôh. Aby bol robot stabilný, nakláňajú telo robota, aby jeho ťažisko ostalo v podpornom mnohouholníku okolo robota.