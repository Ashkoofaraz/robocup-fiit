\section{Používateľská príručka}

\subsection{Predpoklady pre spustenie hráča}
Pre spustenie hráča je potrebné mať správne nainštalovanú jednu z verzií hráča
podľa postupu z inštalačnej príručky. Pred spustením hráča je potrebné aby bol spustený simulačný server na ktorý sa má hráč pripojiť.

\subsection{Nastavenia hráča}

Pre pripojenia hráča na server je potrebné nastaviť adresu a port servera. Tieto je
možné zmeniť v konfiguračnom súbore \texttt{Settings.java}. Je potrebné nastaviť tieto hodnoty:
\begin{itemize}
	\item \texttt{Communication.getInstance().setServerIp()} na 127.0.0.1 - localhost
	\item \texttt{Communication.getInstance().setPort()} na 3100
\end{itemize}

V priložených zdrojových súboroch a tiež vo verzii na GitHub sú tieto hodnoty prednastavené.

\subsection{Spustenie hráča} \label{sec_ap_run}
Spustenie hráča zo zdrojových kódov sa vykoná spustením (najlepšie Eclipse) triedy \texttt{sk.fiit.jim.init.Main}.

\subsection{Zmeny vykonávaných kopov hráča}

Na jednoduchú zmenu vykonávaného pohybu bola už v rámci Tímového projektu vytvorená trieda \texttt{DefaultTactic}. Nami vytvorené pohyby sú prispôsobené nato, aby spolupracovali s aktuálnou implementáciou hráča. Do metódy \texttt{run()} vložíme jeden z kopov, ktorému nastavíme ako vstupný parameter uhol do strany.

Pozíciu hráča je možné zmeniť v triede \texttt{MatchStarterTactic} v metóde \texttt{runBeam()}. Pri výmene pohybov je potrebné hráča opätovne spustiť.