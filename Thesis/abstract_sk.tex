\begin{titlepage}
\large\textbf{Anotácia}\\\\
\normalsize
Slovenská technická univerzita v Bratislave \\
FAKULTA INFORMATIKY A INFORMAČNÝCH TECHNOLÓGIÍ \\
Študijný program: Softvérové inžinierstvo \\
\\
Autor: Bc. Pavol Pidanič \\
Diplomový projekt: Vyššie schopnosti hráča simulovaného robotického futbalu \\
Vedenie diplomového projektu: Ing. Ivan Kapustík \\ 
máj 2014 \\
\\
Práca je venovaná svetovej iniciatíve RoboCup, ktorej cieľom je podpora rozvoja robotiky a umelej inteligencie. Simulovaná robotická 3D liga je časť tejto problematiky, ktorej sa venujú aj výskumníci na Fakulte a informatiky a informačných technológií. Na začiatku opisujeme fakultného hráča s menom JIM a stav riešenia za posledné roky diplomovými prácami a na predmete Tímový projekt. Práca sa zameriava na problém riešenia dynamického kopu. Obsahuje opis riešení najlepších svetových tímov a tímov, ktoré sa venujú hlavne dynamickému a parametrizovateľnému kopu. V popise návrhu uvádzame algoritmus implementácie takého kopu jedného zo svetových tímov a poznatky získané z aktuálneho stavu fakultného hráča súvisiace s návrhom dynamického kopu.

%Práca je venovaná metódam celočíselnej optimalizácie v úlohách prediktívneho riadenia hybridných systémov. Riešenie úloh celočíselnej optimalizácie je zložitý kombinatorický problém, v ktorom počet možných riešení exponenciálne narastá s počtom parametrov optimalizačnej úlohy. V práci je opísaná metóda vetiev a hraníc, ktorej výhodou je redukcia prehľadávaného priestoru prípustných riešení. Implementovaný algoritmus tejto metódy je využitý v prediktívnom riadení hybridných systémov. Algoritmus prediktívneho riadenia využíva hybridný model systému pre predikciu vývoja systému. Na základe získanej predikcie je riešená celočíselná optimalizačná úloha, ktorej výstupom je optimálny riadiaci zásah. Funkcia algoritmu je demonštrovaná na probléme riadenia dynamického systému satelitu. 

\end{titlepage}