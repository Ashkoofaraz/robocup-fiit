\begin{titlepage}
\large\textbf{Anotácia}\\\\
\normalsize
Slovenská technická univerzita v Bratislave \\
FAKULTA INFORMATIKY A INFORMAČNÝCH TECHNOLÓGIÍ \\
Študijný program: Softvérové inžinierstvo \\
\\
Autor: Bc. Pavol Pidanič \\
Diplomový projekt: Vyššie schopnosti hráča simulovaného robotického futbalu \\
Vedenie diplomového projektu: Ing. Ivan Kapustík \\ 
december 2014 \\
\\
Práca je venovaná svetovej iniciatíve RoboCup, ktorej cieľom je podpora rozvoja robotiky a umelej inteligencie. Simulovaná robotická 3D liga je časť tejto problematiky, ktorej sa venujú aj výskumníci na Fakulte a informatiky a informačných technológií. Na začiatku opisujeme fakultného hráča s menom JIM a stav riešenia za posledné roky diplomovými prácami a na predmete Tímový projekt. Práca sa zameriava na problém riešenia dynamického kopu. Riešenie popisuje modul pre inverznú kinematiku, ktorý bol vytvorený za účelom vytvárania dynamických pohybov. Súčasťou je aj ukážka 2 pohybov - pohyb ľavou rukou a ľavou nohou.

%Práca je venovaná svetovej iniciatíve RoboCup, ktorej cieľom je podpora rozvoja robotiky a umelej inteligencie. Simulovaná robotická 3D liga je časť tejto problematiky, ktorej sa venujú aj výskumníci na Fakulte a informatiky a informačných technológií. Na začiatku opisujeme fakultného hráča s menom JIM a stav riešenia za posledné roky diplomovými prácami a na predmete Tímový projekt. Práca sa zameriava na problém riešenia dynamického kopu. Obsahuje opis riešení najlepších svetových tímov a tímov, ktoré sa venujú hlavne dynamickému a parametrizovateľnému kopu. V popise návrhu uvádzame algoritmus implementácie takého kopu jedného zo svetových tímov a poznatky získané z aktuálneho stavu fakultného hráča súvisiace s návrhom dynamického kopu.

\end{titlepage}