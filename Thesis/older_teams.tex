\subsubsection{High5 a Tím 17 žije}

Tieto tímy pracovali na fakultnom agentovi v predošlom akademickom roku ako tím A55Kickers, ktorý je opísaný v kapitole \ref{A55Kickers}.

Tím 17 žije\cite{tim17zije} vytvoril silný priamy kop, ktorý pozostáva z naváženia na jednu nohu, prikročenia druhou nohou, kopu prvou nohou a stabilizovaniu hráča. Pri chôdzi doplnili mierne prikrčenie a zrýchlenie na začiatku pohybu a pri ukončení chôdze jej spomalenie. Ďalej zabezpečili výber pohybov tak, aby sa agent dokázal natáčať a posúvať zároveň. Implementovali metódy pre nájdenie spoluhráčov a určenie, ktorý hráč je najbližšie k lopte. Hráč dokáže predpovedať pozíciu lopty a určiť vhodnosť prihrávky. Doplnili vyhodnotenie herných situácií, či je mužstvo pod tlakom alebo nie. Do testovacieho frameworku pridali grafické zobrazenie hry. V okne je možné zobraziť dáta o jednom alebo všetkých agentoch, o ich polohe a rotácií a informácie o polohe lopty.

Tím High5\cite{high5} vytvoril viacero úrovní kopu špičkou. Vytvorili pohyby pre každú nohu, ktoré sa líšia vo veľkosti náprahu a sile kopnutia. Vylepšili kopnutie bokom a prerobili blokovanie lopty sadnutím. Na zistenie polohy ostatných hráčov upravili parsovanie správ zo see receptora. Vykonali refaktoring testovacieho frameworku. Vyriešili spúšťanie pomocou jednej inicializačnej metódy, upravili logovanie, aby bolo konzistentnejšie a doplnili o konfiguračný súbor. Vytvorili automatické spúšťanie hráča a servera kvôli lepšiemu testovaniu, taktiež testy pohybov, ktoré ohodnotia ich úspešnosť. Ďalej doplnili anotovanie pohybov.