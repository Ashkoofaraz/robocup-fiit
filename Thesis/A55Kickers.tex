\subsection{Tímové projekty}

\subsubsection{A55Kickers}\label{A55Kickers}

Tím A55Kickers \cite{A55Kickers} je jediným tímom v roku 2012, ktorý sa venoval problematike simulovaného futbalu počas predmetu Tímový projekt. 
% Zároveň je aj najaktuálnejším tímom, ktorý odovzdal kompletnú dokumentáciu.\footnote{V akademickom roku 2013/2014 v predmete Tímový projekt sa problematike RoboCup venujú 2 tímy. Počas písania práce prebieha výučba v letnom semestri a predmet nie je uzavretý. Preto sme sa rozhodli, že výsledky tímov Megatroll a Gitmen doplníme, keď tímy odovzdajú celú dokumentáciu s výsledkami, ktoré dosiahli. Aktuálne informácie sú dostupné na oficiálnych stránkach tímov: Megatroll \url{http://team04-13.ucebne.fiit.stuba.sk/} a Gitmen \url{http://team09-13.ucebne.fiit.stuba.sk/}}

Jedným z ich výsledkov je práca na wiki stránke projektu\footnote{Aktuálne dostupná \url{http://labss2.fiit.stuba.sk/TeamProject/2013/team09is-si/wiki/index.html}}, ktorú zdedili od predchádzajúcich tímov High5 a Tím 17 žije. Po doplnení informácií o práci tímu vytvorili novú štruktúru wiki, ktorá podľa ich slov zabezpečuje ľahšiu orientáciu a rýchly prístup k informáciám. Štruktúra sa delí na 3 sekcie. Prvé dve obsahujú informácie o RoboCup iniciatíve a stave projektu na fakulte. Tretia sekcia je určená ako podpora ohľadom inštalačných záležitostí a používateľských príručiek projektu RoboCup na FIIT, ktoré sú nevyhnutné pre riešenie projektu.

Zamerali sa tiež na vylepšenie nižších pohybov (tzv. Low Skills). Vylepšili takmer všetky dovtedy existujúce pohyby od chôdze, presunov po ihrisku a taktiež kopnutia, ktoré stabilizovali a zrýchlili. Ďalej vytvorili nové pohyby - stabilné kopnutia rôznou silou, nové otáčania, úkroky rôznych veľkostí, dokonca aj efektívnejšie vstávania zo zeme. Kvôli prepojeniu vyšších zručností (High Skills), navrhli a implementovali základnú polohu agenta. Základná poloha je taká, do ktorej sa robot nastaví vždy po vykonaní pohybu. Je stabilnejšia a prirodzenejšia, pretože bod ťažiska bol posunutý nižšie. Všetky pohyby premenovali, aby ich pomenovania zodpovedali typu pohybu a ostatným parametrom pohybu. Upravené pohyby využili pri úpravách vyšších pohybov.

Tím upravil už existujúci pohyb \texttt{Kick}, ktorý by zodpovedal prihrávaniu. Pohybu doplnili parameter, ktorý slúži na zadanie miesta, na ktoré je potrebné kopnúť. Na výber sily kopu využili už existujúcu metódu pre určenie vzdialenosti lopty od požadovaného bodu.



%V závere naznačujú, kam by sa projekt na fakulte mohol ďalej uberať, %pretože robot nedosahuje vlastnosti hráčov svetovej úrovne. 
%------------------------------------------
%Z tohto dôvodu je do budúcnosti potrebné pracovať na zdokonaľovaní taktiky, vylepšovaní videnia sveta a v neposlednom rade aj na samotných pohyboch a stabilizácii agenta. Keďže súčasný fakultný agent ma implementované len statické pohyby je možne ho rozšíriť o dynamické pohyby, ktoré by mali zvýšiť rýchlosť agenta a tým zvýšiť jeho konkurencie schopnosť.
%Ďalším krokom by mohla byť stabilizácia agenta s využitím stabilizačných kritérií, ktoré by zabezpečili zníženie pádov pri hraní futbalu. Tiež sa dá pracovať na zdokonaľovaní taktiky, ktorá zatiaľ nie je v takom stave, aby odzrkadľovala hranie reálneho futbalu. V taktike sa dá pracovať na tvorbe prihrávok, nabiehaní na prihrávky, obchádzaní súpera a rôznych iných možnostiach
%----------------------------------------