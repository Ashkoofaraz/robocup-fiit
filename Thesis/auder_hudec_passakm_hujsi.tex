\subsubsection{Staršie diplomové práce}\label{sec_auder_hudec}

V tejto časti si popíšeme výsledky starších prác na FIIT ako opísané diplomové práce Lukáša Ďurčáka (kapitola \ref{sec_durcak}) a Petra Paššáka (kapitola \ref{sec_passakp}) pridané do implementácie robota JIM (kapitola \ref{jim}). V niektorých prácach dosiahli zrýchlenie niektorých pohybov. Avšak tieto výsledky nebudeme uvádzať, pretože pohyby boli optimalizované v práci Petra Paššáka (kapitola \ref{sec_passakp}).

Miloš Auder vo svojej práci\cite{auder} z roku 2012 skúmal prístupy ku chôdzi, spôsoby hľadania cesty a tiež metódy ako genetické algoritmy a strojové učenie a ich následné využitie v doméne Robocup. Vytvoril nové otáčanie hráča na mieste, ktoré je možné parametrizovať a tak dosiahnuť väčší uhol natočenia. Vznikol aj nový typ chôdze s tzv. ľudským výzorom. Avšak efektívna bola len na kratšie vzdialenosti. Pri hľadaní cesty vznikli 2 prístupy. Jedným je priblíženie hráča k lopte. A pri druhom prístupe, v ktorom sa hráč snaží prejsť oblasť so súperovými hráčmi, použil genetický algoritmus. Výsledkom sú body, cez ktoré má hráč prejsť.

Ďalším autorom v roku 2012 je Ján Hudec\cite{hudec}. Zameral sa na stabilizačné funkcie a ovládače pre chôdzu. Pri stabilizácii využil hodnoty Zero Moment Point a informácie o pozícii tela robota zapracovaním vyhodnocovania Force Resistance (FR) perceptora.

Poslednou diplomovou prácou v roku 2012 je práca Martina Paššáka. Tá úzko súvisí s prácou jedného z autorov hráča JIM - Ivana Hujsiho. Obaja sa venovali pohybom robota. 

Ivan Hujsi\cite{Hujsi} navrhol a vytvoril 19 nových pohybov. Z toho bolo 11 kopov určených na prihrávanie a streľbu pri rôznych pozíciách lopty. Vznikli pohyby na posúvanie a napravovanie lopty a tiež niekoľko základných pohybov ako chôdza, vstávanie z brucha a chrbta. Tieto pohyby boli vytvorené ručne bez použitia automatických nástrojov. Tiež navrhol a implementoval algoritmus výberu smeru kopnutia.

Už z názvu práce Martina Paššáka\cite{passak_martin} vyplýva, že sa zameral na spôsoby, ako hráč manipuluje s loptou. Vo výsledkoch vylepšil zatáčajúcu chôdzu, pri ktorej robot dynamicky upravuje smer chôdze zmenou kroku. Bol pridaný modul, s ktorým je možné zrýchľovať a spomaľovať chôdzu. Upravil približovanie k lopte a taktiež upravil obchádzanie lopty. Vytvoril chôdzu s loptou na určené miesto. Pri kopaní vytvoril nové pohyby a model, ktorým je možné kopy parametrizovať. Na záver prepracoval hlavný plán hráča, ktorý využíva rozhodovací strom. Tým dosiahol jednoduchšie rozhodovanie hráča.