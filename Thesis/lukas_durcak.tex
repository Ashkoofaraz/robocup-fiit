\subsubsection{Diplomová práca Lukáša Ďurčáka} \label{sec_durcak}

Diplomová práca Lukáša Ďurčáka\cite{durcak} je jednou z dvoch posledných dokončených prác, ktoré boli ukončené v roku 2013.

Autor sa v nej venoval rozšíreniu vyšších schopností hráča - rozhodovanie a zhodnotenie situácie na ihrisku. Pri vyšších schopnostiach sa zameral na schopnosť agenta odhaliť kolízie medzi hráčom a prekážkou, alebo hráčom a ďalším hráčom a vytvoril systém, ako týmto situáciám predísť. Hráč je schopný obchádzať protihráčov a ostatné prekážky a tak nebude dochádzať k neželaným pádom. Pri riešení kolízií sa rozhodol využiť analytickú geometriu. Aj keď zápasy simulovanej ligy prebiehajú v trojrozmernom priestore, každá ním riešená situácia sa dá pretransformovať do jednoduchších útvarov v dvojrozmernom priestore. Do modelu kolízií je zarátaná aj predikcia pohybu hráčov. Táto schopnosť dokáže určiť približný smer a rýchlosť pohybu ostatných hráčov na ihrisku. 

Vlastnosťou, ktorá súvisí so zameraním práce, je rozhodovanie výberu miesta na kopnutie, teda prihrávku. Hráč sa rozhoduje, či kopne loptu smerom na bránu, pokúsi sa prihrať alebo bude driblovať na vhodnejšie miesto. Hráč vykoná kop na bránu, ak sa medzi ním a bránou nenachádza žiadna prekážka a je v dostatočnej vzdialenosti od brány. Dostatočná vzdialenosť je taká, ak je menšia ako vzdialenosť, ktorú je možno dosiahnuť kopom. Pri kopaní na bránu sa nesústredí na stred brány, ale berie do úvahy aj postavenie súperovho brankára. Prihráva v takom prípade, keď je vzdialenosť od brány väčšia ako dokáže kopnúť a ak existuje prekážka, ktoré bráni kopnutiu.

Hráča rozšíril o schopnosti tímovej hry. Jednou zo schopností je komunikácia medzi hráčmi. Hráči si dokážu medzi sebou posielať správy a získavať z nich informácie. Ďalšou doplnenou schopnosťou je spôsob hrania vo formáciách. Vytvoril a otestoval niekoľko rôznych formácií. Formácie rozlišujú defenzívnu alebo ofenzívnu pozíciu hráča.

Nakoniec vytvoril testovací nástroj. Umožňuje vytváranie herných situácií. Agent sa môže na ihrisku správať na základe umelo vytvorenej hernej schémy. Slúži na lepšie pochopenie činností a rozhodnutí, ktoré hráč vykonáva pomocou dostupných informácií z prostredia.