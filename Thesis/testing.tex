\section{Overenie riešenia}

Implementáciu sme sa pokúsili vyskúšať na jednoduchej situácii. Do plánovača sme doplnili novú triedu a súčasne vytvorili nový pohyb, ktorý je vložený do plánovača. Trieda pre pohyb vytvorí sekvenciu koncových bodov a natočení v priestore pre koncový efektor. Tieto sekvencie necháme vykonať na simulačnom prostredí.

Vytvorili sme 2 pohyby - jeden pre ľavú ruku a jeden pre ľavú nohu. Pokus s ľavou nohou sa vykonáva po páde robota. Aktuálna implementácia neberie do úvahy stabilitu robota. Pri pohybe nohou by spadol, aj keby začal vo vzpriamenej polohe. Pokusné body je možné meniť v Ruby triede \texttt{RubyDynamicMove}. Avšak musia byť dosiahnuteľné natočením kĺbov. 

Videá sú dostupné na priloženom médiu (viď príloha \ref{appendix_medium}).